\documentclass[11pt]{amsart}
\usepackage{geometry}                % See geometry.pdf to learn the layout options. There are lots.
\geometry{letterpaper}                   % ... or a4paper or a5paper or ... 
%\geometry{landscape}                % Activate for for rotated page geometry
%\usepackage[parfill]{parskip}    % Activate to begin paragraphs with an empty line rather than an indent
\usepackage{graphicx}
\usepackage{amssymb}
\usepackage{epstopdf}
\usepackage{../packages/bnf}
\usepackage{../packages/elogic}
\DeclareGraphicsRule{.tif}{png}{.png}{`convert #1 `dirname #1`/`basename #1 .tif`.png}

\title{About Custom Operators}
\author{Simon}
%\date{}                                           % Activate to display a given date or no date

\begin{document}
\maketitle
%\section{}
%\subsection{}

This essay is aimed at developing an algorithm for parsing expressions in a context where it is possible to define custom operators along with their precedences.  Since we can't name the operators, we have to use a technique which is not commonly use for defining the syntax of languages.  As far as I know, the technique did not exist prior to the redaction of this essay.  We have to use parameterized production rules.  

\begin{grammar} 
		[(colon){$\rightarrow$}] 
		[(semicolon){ $\\ \quad | \quad$}] 
		[(period){\\}] 
		[(quote){�}{�}] 
	expr (oper)  : expr (prec"."op) op expr (prec"."op)  \\
	$\qquad$ with $op \in oper$
	; 'word'
\end{grammar}

\end{document}  