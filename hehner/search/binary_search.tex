\documentclass[11pt]{amsart}
\usepackage{geometry}                % See geometry.pdf to learn the layout options. There are lots.
\geometry{letterpaper}                   % ... or a4paper or a5paper or ... 
%\geometry{landscape}                % Activate for for rotated page geometry
%\usepackage[parfill]{parskip}    % Activate to begin paragraphs with an empty line rather than an indent
\usepackage{graphicx}
\usepackage{amssymb}
\usepackage{epstopdf}
\usepackage{algorithmic}
\usepackage{ifthen}
\usepackage{../packages/elogic}
\usepackage{../packages/bsymb}
\DeclareGraphicsRule{.tif}{png}{.png}{`convert #1 `dirname #1`/`basename #1 .tif`.png}

\title{Binary Search with Hehner's Calculus}
\author{Simon Hudon}
\date{}                                           % Activate to display a given date or no date

\begin{document}
\maketitle

\section{Introdution}

This is an exercise I did to practice with Hehner's program calculus.  The problem is a classical one but, still, it's solution is easy to get wrong.

\section{Notation}

The notation I use is a slightly revised version of that used in Hehner's book.  It consists of the elementary structured programming construct but, instead of using a recursive definition for $while$ I use an $until$ loop defined in the usual way.  

The calculus works with predicates and relations over state spaces.  We will use $\Sigma$ as the type for our spaces.  In this context, we define a program as a relation between two states that we can interpret as the state before the execution of the program and the state after.  The type, given in curried form, is therefore $\Sigma \tfun \Sigma \tfun \mathbb{B}$. 

\note{To be continued}

\section{Problem}

We have a sorted array $A(0 \le i<n)$ in which we are looking for $x$.  We make the following assumptions:

\begin{align} 
	& A.0 \le x \\
	& x < A.n \\
	& \iter{\forall}{i,j}{i \le j}{A.i \le A.j}
\end{align}

We want to compute $i$ such that the following predicates hold:

\begin{align*}
	& Q1: \quad 0 \le i \land i < n \\
	& Q2: \quad A.i \le x \land x < A.(i+1)
\end{align*}

\section{Solution}

Since we have to look at an unknown number of places for x, we start right away with designing a loop that solves the problem.  We use the technique of uncoupling to find our loop invariant from the postcondition $Q2$: since $i$ appears twice in the postcondition, we can replace an expression containing one occurrence by an expression of $j$ and build our loop around the idea of bringing $i$ and $j$ close enough to one another.

\begin{deriv}
		\step{Q2}
			\infer{\implied}{Leibniz}
		\step{A.i \le x \land x < A.j \land j = i + 1}
			\infer{=}{Define $J1 \equiv A.i \le x \land x < A.j$ and $G1 \equiv j = i + 1$}
		\step{J1\land G1}
\end{deriv}

We keep $Q1$ as an invariant and also take its equivalent in terms of $j$.

\begin{align*}
	& J2: \quad 0 \le i \land i < n  \\
	& J3: \quad 0 < j \le n 
\end{align*}

Contrarily to usual practice, we will start our derivation by refining the requirement for the loop to terminate.  The usual technique for this is to identify a decreasing value, the variant.  Since we are trying to bring $i$ and $j$ closer and closer to one another, their difference is a good idea for a variant.  To simplify the whole thing, we'll have $i$ approach from the lower values and $j$ from the higher values.

\[
	J4: \quad i < j
\]

\begin{deriv}
	\step{j' - i' < j - i}
		\infer{=}{J4 and G1}
	\step{j' - i' < j - i \land \iter{\exists}{k}{}{i < k \land k < j}}
		\infer{=}{Distributivity of $\lor$ over $\exists$}
	\step{\iter{\exists}{k}{}{j' - i' < j - i \land i < k \land k < j}}
		\infer{\implied}{Strengthening}
	\step{\iter{\exists}{k}{}{j' - i' < j - i \land i < k \land k < j \land ((i' = k \land j' = j) \lor (i' = i \land j' = k))}}
		\infer{}{}
\end{deriv}

\end{document}  