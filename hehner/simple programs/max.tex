\documentclass[11pt]{amsart}
\usepackage{geometry}                % See geometry.pdf to learn the layout options. There are lots.
\geometry{letterpaper}                   % ... or a4paper or a5paper or ... 
%\geometry{landscape}                % Activate for for rotated page geometry
%\usepackage[parfill]{parskip}    % Activate to begin paragraphs with an empty line rather than an indent
\usepackage{graphicx}
\usepackage{amssymb}
\usepackage{epstopdf}
\usepackage{algorithmic}
\usepackage{tocloft}
\usepackage{ifthen}
\usepackage{../../packages/elogic}
\DeclareGraphicsRule{.tif}{png}{.png}{`convert #1 `dirname #1`/`basename #1 .tif`.png}

\title{Derivation of a program to calculate the max of two numbers}
\author{Simon Hudon}
\date{}                                           % Activate to display a given date or no date

\renewcommand{\max}{\uparrow}
\renewcommand{\[}{\begin{align}}
\renewcommand{\]}{\end{align}}
%\renewcommand{\ref}[1]{\eqref{#1}}
\newcommand{\predicate}[2]{  \begin{equation*} #1: \qquad #2 \end{equation*} }
\newcommand{\axiom}[2]{ \begin{equation} #1 \label{2} \end{equation} }

\newcommand{\kw}[1]{\textbf{  #1  }}
\newcommand{\select}[1]{ & \kw{if} #1 \\ & \kw{fi} }
\newcommand{\choice}{ \\ & \kw{or}}
\newcommand{\then}{\kw{then} \\ & \qquad }
\usepackage{listings}

\begin{document}
\maketitle

%\section{Introdution}
This is a simple exercise to get started with Hehner's calculus.

We use $\max$ to denote the function that denotes the maximum of two numbers.  We want to implement:

\predicate{Q1}{z = x \max y}

The first thing that might be of use is two know that $x \max y$ can only take two values.

\axiom{x = x \max y \lor y = x \max y}{0}

\begin{deriv}
	\step{Q1 [z := z']}
		\infer{=}{$\eqref{0}$}
	\step{z' = x \max y \land (x = x \max y \lor y = x \max y)}
		\infer{=}{Distributivity of $\land$ over $\lor$ }
	\step{(z' = x \max y \land x = x \max y) \lor (z' = x \max y \land y = x \max y)}
		\infer{=}{Leibniz twice}
	\step{(z' = x \land x = x \max y) \lor (z' = y \land y = x \max y)}
\end{deriv}
	
The last step of the calculation gives us the shape of an acceptable program.
\lstset{language=Eiffel}
\begin{lstlisting}
P1:	if $x = x \max y$ then
		z := x
	or $y = x \max y$ then
		z := y
	end
\end{lstlisting}

This program, even if it gives a direction to the development does not do much more than beg the question: if we could evaluate $x = x \max y$ directly, we could simply use the mechanism for assigning z directly.  However, a feature of lattice theory tells us a very interesting fact:

\axiom{x = x \max y \equiv y \le x}{1}

Combined with the following, it can make $P1$ executable.

\axiom{x \max y = y \max x}{2}

\begin{deriv}
	\step{P1}
		\infer{=}{\eqref{2}}
	\step{(z' = x \land \"x = x \max y\") \lor (z' = y \land \" \*y = y \max x\* \")}
		\infer{=}{\eqref{1} twice}
	\step{(z' = x \land \* x \le y \* ) \lor (z' = y \land \* y \le x \*)}
\end{deriv}

Which yields the more interesting:

\begin{lstlisting}
P2:	if $x \le y$ then
		z := x
	or $y \le x$ then
		z := y
	end
\end{lstlisting}



\end{document}  